\section*{Midterm 1}

A note on the third question, part (b). No one really answered what it was asking for, so we have to develop this idea more.

There is a condition where $R \ge 0, B \ge -1$, but the logic still applies? No. the utility is maximized when you simply only buy $B$, and not $R$. Basically test for the end points, where it's all $R$ and all $B$, as the critical point isn't necessarily the end point.

When we talk about things not being well-defined, it deals with the utility function. An indifference curve of complements (L-shaped) would not be well-defined because we can't find the slope of that thing (think of absolute value curves). If we think about the Lagrangian mechanically, we're taking partial derivatives that should work, but it would not work in the case that I just outlined.

For the last part of $U = RB^2$, this is well-defined, and if you test the end points, you would unsurprisingly find that the critical point is the right bundle.