\section*{Important Application of Matrices: Economic Systems}
Suppose we have exogeneous variables Income, Consumption, and Government Expenditure.
\begin{equation*}
    \begin{pmatrix}
    1  & -1 & -1 \\
    -b & 1 & 0 \\
    -g & 0 & -1
\end{pmatrix}
\cdot
\begin{pmatrix}
    Y \\
    C \\
    G
\end{pmatrix} = 
\begin{pmatrix}
    I_0 \\
    a-bT_0\\
    0
\end{pmatrix}
\end{equation*}
Where the variables $I_0, T_0, a, b$ are policy variables that we're interested in solving/achieving.

We can see that $Y = I_0 + C + G$ based on the first row. The variable $b$ can be described as marginal propensity to consume. Using Cramer's rule, you can actually create some $Y', C', G'$ that depend on other variables that we specified beforehand.


